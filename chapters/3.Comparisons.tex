\begin{frame}{Größenordnungen}
	\begin{itemize}
		\item DB Fahrplan 1996/97 (\textbf{nur} Züge)\blfootnote{(Pyrga et al. \textit{Time-Expanded vs Time-Dependent Models for Timetable Information})}:
	\end{itemize}
	
	\begin{center}
		\begin{tabular}{ c|r|r } 
			& Time Expanded & Time Dependent \\
 			\hline
 			\hline
 			Knoten & 931746 & 6961 \\
 			Kanten & 1397619 & 18664 \\
 			\hline
		\end{tabular}
	\end{center}
	
	\vspace{3em}	
	
	\begin{center}
		\begin{tabular}{ l|c|c } 
			& Time Expanded & Time Dependent \\
 			\hline
 			\hline
 			Algorithmus & Dijkstra & Dijkstra + Binärsuche \\
 			$\varnothing{}$ Laufzeit & 44.17ms & 5.61ms \\
 			Besuchte Knoten & 33653 & 1515 \\
 			\hline
		\end{tabular}
	\end{center}
\end{frame}


\begin{frame}{Performanz}
	\begin{itemize}
		\item Aber: Komplexität von TD wächst schnell mit mehr Kriteria und Regeln\blfootnote{(Pyrga et al.: \textit{Experimental Comparison of Shortest Path Approaches for Timetable Information}, 2004)}
		\item Also in realistischen Szenarien!
		\begin{itemize}
			\item TD nur $58\%$ schneller (in CPU-Zeit) als TE
		\end{itemize}
	\end{itemize}

\end{frame}

